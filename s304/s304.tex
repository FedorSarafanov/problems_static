\documentclass[a4paper,14pt]{extarticle}
\def\source{/home/osabio/tex/templates}
\input{\source/head.tex}
\yakovlev{304}{Магнитостатика}
% Стыдить лжеца, шутить над дураком
% И спорить с женщиной — всё то же,
% Что черпать воду решетом:
% От сих троих избавь нас, боже!..

% М. Ю. Лермонтов.
\usetikzlibrary{decorations.markings}
\usepackage[outline]{contour}
\contourlength{4pt}
\begin{document}

\begin{figure}[H]
    % \centering
    \begin{tikzpicture}
    \begin{scope}[yshift=2cm]
        \lineann[2]{0}{-2}{$a$}
    \end{scope}

    \begin{scope}[
        very thick,decoration={
        markings,
        mark=at position 0.5 with {\arrow{>}}}
        ] 
        \draw[postaction={decorate}] (0,4)--(0,0);
        \draw[axis,->] (0,3)--(0,4) node[above] {$+y$};
        \draw[axis,->] (-3,0)--(-4,0) node[left] {$+x$};
        \draw[axis,->] (0,0,-7)--(0,0,-8) node[below, xshift=1em] {$+z$};
        \draw[postaction={decorate}] (0,0)--(-4,0);
        \draw[dashed] 
            (-4,4) node[above] {$O$}
            -- (0,0) node[anchor=west] {$A$}
            -- (4,-4) node[right] {$O'$};

        \draw[dashed] 
            (4,4) node[above] {$B$}
            -- (-4,-4) node[right] {$B'$};            

        \draw (135:{sqrt(2)}) node [] {\contour{white}{$\bigotimes$}}  node[above, yshift=0.5em] {$d\vec{B}_1$};
        \draw (135:2.5) node [] {\contour{white}{$\bigotimes$}} node[above, yshift=0.5em] {$d\vec{B}_2$};

        % \draw (45:{sqrt(2)}) node [] {\contour{white}{$\bigodot$}}  node[above, yshift=0.5em] {$d\vec{B}_1$};
        % \draw (45:2.5) node [] {\contour{white}{$\bigotimes$}} node[above, yshift=0.5em] {$d\vec{B}_2$};     

        % \draw (135+90:{sqrt(2)}) node [] {\contour{white}{$\bigotimes$}}  node[above, yshift=0.5em] {$d\vec{B}_1$};
        % \draw (135+90:2.5) node [] {\contour{white}{$\bigodot$}} node[above, yshift=0.5em] {$d\vec{B}_2$};

        \draw (45-90:{sqrt(2)}) node [] {\contour{white}{$\bigodot$}}  node[above, yshift=0.5em] {$d\vec{B}_1$};
        \draw (45-90:2.5) node [] {\contour{white}{$\bigodot$}} node[above, yshift=0.5em] {$d\vec{B}_2$};   

        \draw (5,0) -- (14,0);          
        \draw[] (9,-4) -- ++(0,8);    

        \begin{scope}[xshift=8cm]
        \draw[scale=2,domain=0.031:1.22,smooth,variable=\x,blue] plot ({\x},{sqrt(0.2-0.6*(\x^0.9-0.62)^2)});
        \draw[scale=2,domain=0.031:1.22,smooth,variable=\x,blue] plot ({\x},{-sqrt(0.2-0.6*(\x^0.9-0.62)^2)});    
        \end{scope}
        \begin{scope}[xshift=7.5cm]
        \draw[scale=4,domain=0.031:1.22,smooth,variable=\x,blue] plot ({\x},{sqrt(0.2-0.6*(\x^0.9-0.62)^2)});
        \draw[scale=4,domain=0.031:1.22,smooth,variable=\x,blue] plot ({\x},{-sqrt(0.2-0.6*(\x^0.9-0.62)^2)});    
        \end{scope}        
        % y^2=       


        \draw[->] (0,0,-4) -- ++(1,0,0) node [above] {$d\vec{B}_1$};
        \draw[->] (0,0,-4) -- ++(0,-1,0) node [right] {$d\vec{B}_2$};

        \draw[->] (0,0,4) -- ++(-1,0,0) node [left] {$d\vec{B}_1$};
        \draw[->] (0,0,4) -- ++(0,1,0) node [left] {$d\vec{B}_2$};        


    \end{scope}    
    \end{tikzpicture}
\end{figure}

Рассмотрим поле вертикального участка провода. 

Используем ранее выведенную формулу для произвольного отрезка проводника с током $B^p=\frac{\beta I}{a}[\sin\alpha_2-\sin\alpha_1]$, где $\alpha_1=\pm\frac{\pi}{4}$, $\alpha_2=\frac{\pi}{2}$, а из постороения видно, что если $r$ -- координата на $AO$ с нулем в $A$, то $a(r)=r\sin\frac{\pi}{4}$. 

Тогда поле вертикального участка провода даст
\begin{equation*}
    B_1=\frac{\sqrt{2}\beta I}{r}[1\pm\frac{1}{\sqrt{2}}]
\end{equation*}

Покрутив векторное произведение, легко увидеть, что горизонтальный участок провода даст такое же по направлению поле, а по модулю можно воспользоваться симметрией задачи. Тогда полное поле

\begin{equation*}
    B=\frac{2\beta I}{r}[\sqrt{2}\pm1]
\end{equation*}

Плюс - минус здесь появился из $\sin\alpha_1$ и при <<+>> соответствует углу 
$-\frac{\pi}{4}$, т.е. координате на $AO$, а при <<->> на $AO'$ соответственно.

На оси $B-B'$ $a(r)=r$:
\begin{equation}
    B_1=\frac{BI}{a},\quad
    B_2=\frac{BI}{a}
\end{equation}
Тогда модуль поля будет
\begin{equation*}
    B=\sqrt{\left(\frac{BI}{a}\right)^2+\left(\frac{BI}{a}\right)^2}=
        \frac{\sqrt{2}\beta I}{r}
\end{equation*}


\end{document}