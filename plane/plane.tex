\documentclass[a4paper,14pt]{extarticle}
\def\source{/home/osabio/tex/templates}
\input{\source/head.tex}
\wrote{Пластины}{Электростатика}

\begin{document}
0) Заряжены пластины A и B.
\begin{tikzpict}
	\draw (0,0) node[below] {A} -- ++(0,4) node[right, pos=0.5] {$+q$};
	\draw (8,0) node[below] {B} -- ++(0,4) node[left, pos=0.5] {$-q$};
	\lineann[-1.5]{0}{8}{$d_0$};
\end{tikzpict}

1) Вставили пластины $C$ и $D$

\begin{tikzpict}
	\draw (0,0) node[below] {A} -- ++(0,4) node[right, pos=0.5] {$+q$};

	\draw[interface] (2,0) rectangle ++(1,4);
	\draw (2,0) node[below] {C$_A$} -- ++(0,4) node[left, pos=0.5] {$-q_i$};
	\draw (3,0) node[below] {C$_D$} -- ++(0,4) node[right, pos=0.5] {$+q_i$};

	\draw[interface] (5,0) rectangle ++(1,4);
	\draw (5,0) node[below] {D$_C$} -- ++(0,4) node[left, pos=0.5] {$-q_i$};
	\draw (6,0) node[below] {D$_B$} -- ++(0,4) node[right, pos=0.5] {$+q_i$};


	\draw (8,0) node[below] {B} -- ++(0,4) node[left, pos=0.5] {$-q$};
	% \lineann[-1.5]{0}{8}{$d_0$};
\end{tikzpict}

2) Замкнули $C$ и $D$

\begin{tikzpict}
	\draw (0,0) node[below] {A} -- ++(0,4) node[right, pos=0.5] {$+q$};

	\draw[interface] (2,0) rectangle ++(1,4);
	\draw (2,0) node[below] {C$_A$} -- ++(0,4) node[left, pos=0.5] {$-q_i$};
	\draw (3,0) node[below] {C$_D$} -- ++(0,4);% node[right, pos=0.5] {$+q_i$};

	\draw[interface] (5,0) rectangle ++(1,4);
	\draw (5,0) node[below] {D$_C$} -- ++(0,4);% node[left, pos=0.5] {$-q_i$};
	\draw (6,0) node[below] {D$_B$} -- ++(0,4) node[right, pos=0.5] {$+q_i$};

	\draw (3,2) -- (5,2);

	\draw (8,0) node[below] {B} -- ++(0,4) node[left, pos=0.5] {$-q$};
	% \lineann[-1.5]{0}{8}{$d_0$};
\end{tikzpict}
Можно представить как два последовательно соединенных конденсатора. Энергия такой системы будет
\begin{equation}
	W_2=\frac{q^2}{2C_{a-c}}+\frac{q^2}{2C_{d-b}}
\end{equation}
где
\begin{equation}
	C_{a-c}=\frac{S}{k\cdot4\pi x}, \quad
	C_{d-b}=\frac{S}{k\cdot4\pi (d_0-x-d)}
\end{equation}
Тогда
\begin{equation}
	W_2=\frac{q^2}{2S}\cdot k\cdot 4\pi (d_0-d)
\end{equation}

3) Разомкнули $C$ и $D$ и поменяли местами

\begin{tikzpict}
	\draw (0,0) node[below] {A} -- ++(0,4) node[right, pos=0.5] {$+q$};

	\draw[interface] (2,0) rectangle ++(1,4);
	\draw (2,0) node[below] {D$_A$} -- ++(0,4) node[left, pos=0.5] {$-q_i$};
	\draw (3,0) node[below] {D$_C$} -- ++(0,4) node[right, pos=0.33] {$+q_i$} node[right, pos=0.66] {$+q$};

	\draw[interface] (5,0) rectangle ++(1,4);
	\draw (5,0) node[below] {C$_D$} -- ++(0,4) node[left, pos=0.33] {$-q_i$}  node[left, pos=0.66] {$-q$};
	\draw (6,0) node[below] {C$_B$} -- ++(0,4) node[right, pos=0.5] {$+q_i$};


	\draw (8,0) node[below] {B} -- ++(0,4) node[left, pos=0.5] {$-q$};
	% \lineann[-1.5]{0}{8}{$d_0$};
\end{tikzpict}
После соединения на пластинах $C$ и $D$ с внутренней стороны сосредотачивался заряд $+q$ и $-q$ соотвественно. 

После разъединения и переворота там еще должно произойти разделение индуцированных зарядов. При этом индуцированный заряд $q_i$ должен быть равен $q$, но тогда на внутренних сторонах $D_C$ и $C_D$ будут заряды $+2q$ и $-2q$ соответственно.

Можно найти энергию эквивалентной системы:
\begin{figure}[H]
    \centering
	\begin{circuitikz}
		\draw (0,0)
		to[] ++(0,2)
		to[C=$C_{AD_A}$] ++(2,0)
		to[C=$C_{D_CC_D}$] ++(0,-2)
		to[] ++(-2,0)
		to[] ++ (5,0)
		to[] ++ (0,2);
		\draw (2,2)
		to[C=$C_{D_AB}$] ++(3,0);
		% to[battery1=$\mathscr{E}$] ++(0,-2)		
		% to[] ++(-2,0);
	\end{circuitikz}
\end{figure}

\begin{equation}
	W_3=\frac{q^2}{2C_{AD_A}}+
	\frac{4q^2}{2C_{D_CC_D}}+
	\frac{q^2}{2C_{D_BB}}=\frac{4\pi k q^2}{2S}(x+4d+d_0-x-d)=\frac{4\pi k q^2}{2S}(3d+d_0)
\end{equation}

Тогда
\begin{equation}
	A=W_3-W_2=\frac{4\pi k q^2}{2S}(3d+d_0)-
	\frac{q^2}{2S}\cdot k\cdot 4\pi (d_0-d)=
	\frac{k4\pi  q^2}{2S}\cdot 4d=\frac{k\pi8q^2}{S}
\end{equation}

\end{document}