\documentclass[a4paper,14pt]{extarticle}
\def\source{/home/osabio/tex/templates}
\input{\source/head.tex}
\wrote{Шар}{Электростатика}

\begin{document}

\begin{tikzpict}
	\draw[th] (0,0) circle (2cm);
	\draw[dashed] (0,0) circle (1.8cm);
	\draw[draw=none] (-5,0) (5,0); % Выравнивание по центру
	\draw[axis,->] (0,0) -- (4,0) node[right] {$+r$};
	\foreach \i in {0,10,...,360}{
		\draw[magenta,->] (\i:3) -- (\i:1.5);
	}
\end{tikzpict}
Пусть дано распределение потенциала 
\begin{equation}
	\phi(r)=ar^2+b
\end{equation}
Тогда
\begin{equation}
	E(r)=-\diffp{\phi}{r}=-2ar
\end{equation}
C одной стороны,
\begin{align}
	\oiint\limits_{(S)}\vec{E}\,d\vec{S}=\label{eq:3}
	k\cdot 4\pi q_{in}=&
	k\cdot 4\pi \iiint\limits_{(V)}\rho dV=
	k\cdot 4\pi \int\limits_{(r)} \rho(r)\cdot 4\pi r^2\, dr=\\
	=& 
	k\cdot 16\pi^2 \int\limits_{(r)} \rho(r)r^2\,dr
\end{align}
%Где $\rho=\rho(r)$ из соображений симметрии.
Если расписывать сам интеграл (\ref{eq:3}):
\begin{equation}
	\oiint\limits_{(S)}\vec{E}\,d\vec{S}=
	-2ar\cdot 4\pi r^2
\end{equation}
Тогда
\begin{equation}
	-2ar\cdot 4\pi r^2=k\cdot 16\pi^2 \int\limits_{(r)} \rho(r)r^2\,dr \quad\bigg| \frac{d}{dr}
\end{equation}
\begin{equation}
	-6ar^2=k\cdot 4\pi \rho(r)r^2
\end{equation}
Откуда окончательно
\begin{equation}
	\rho=-\frac{3a}{2k\pi}
\end{equation}
\end{document}