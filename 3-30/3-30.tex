\documentclass[a5paper,10pt]{article}\usepackage[usenames,dvipsnames]{color}\usepackage{extsizes,cmap,graphicx,misccorr,indentfirst,makecell,multirow,ulem,geometry,amssymb,amsfonts,amsmath,amsthm,titlesec,float,fancyhdr,wrapfig,tikz,pgfplots}\usepackage[T2A]{fontenc}\usepackage[utf8x]{inputenc}\usepackage[english, russian]{babel}\usetikzlibrary{decorations.pathreplacing,decorations.pathmorphing,patterns,calc,scopes,arrows,through,positioning,shapes.misc}\graphicspath{{img/}}\linespread{1.3}\frenchspacing\geometry{left=1cm, right=1cm, top=2cm, bottom=1cm, bindingoffset=0cm}\pagestyle{fancy}\fancyhead{}\fancyhead[R]{Сарафанов Ф.Г.} 
\fancyhead[C]{Электростатика}
\fancyhead[L]{Иродов -- №3.30} 
\fancyfoot{}
\renewcommand{\labelenumii}{\theenumii)}
\tikzset{
	force/.style={>=latex,draw=blue,fill=blue,>=triangle 45},
    axis/.style={densely dashed,blue},
    interface1/.style={draw=gray!60,.
        postaction={draw=gray!60,decorate,decoration={border,angle=-135,
        amplitude=0.3cm,segment length=2mm}}},
    interface/.style={
        pattern = north east lines,
        draw    = none,
        pattern color=gray!60,          
    },
    plank/.style={
        fill=black!60, 
        draw=black,
        minimum width=3cm,
        inner ysep=0.1cm,
        outer sep=0pt,
        yshift=0.75cm,
        pattern = north east lines,
        pattern color=gray!60, 
    },
    cargo/.style={
        rectangle,
        fill=black!70,              
        inner sep=2.5mm,
    }	
}

% Draw line annotation
% Input:
%   #1 Line offset (optional)
%   #2 Line angle
%   #3 Line length
%   #5 Line label
% Example:
%   \lineann[1]{30}{2}{$L_1$}
\newcommand{\lineann}[4][0.5]{%
    \begin{scope}[rotate=#2, blue,inner sep=2pt]
        \draw[dashed, blue!40] (0,0) -- +(0,#1)
            node [coordinate, near end] (a) {};
        \draw[dashed, blue!40] (#3,0) -- +(0,#1)
            node [coordinate, near end] (b) {};
        \draw[|<->|] (a) -- node[fill=white] {#4} (b);
    \end{scope}
}
\begin{document}

\begin{figure}[H]
    \centering
    \begin{tikzpicture}
        % \draw (0,0) node[left]{$\phi_0$} circle (1cm)
        %     circle (2cm);
        % \draw[fill=black] (0,0) circle (2pt);
        % \draw[axis,->] (0,0) -- (3,0) node[right] {$+r$};
        % %
        % \lineann[4]{90}{2}{$R_2$};
        % \lineann[3]{90}{1}{$R_1$};
        \fill[magenta!10] (0,0) circle (2);
        \fill[white] (0,0) circle (1.75);
        \draw[black,thick] (0,0) circle (2);
        \foreach \i in {1.75,2}{
            \draw[black!50] (0,0) circle (\i);
        };
        \foreach \angle in {0,10,...,360}{
            \draw[black,thick] (0,0) ++ (\angle:1.75) -- ++ (\angle:0.25);
        };

        \draw[thick] (3,0) ++ (0,-2) rectangle ++(0.1,4);
        \draw[dashed, blue,->] (3,0) -- (8,0) node[thick,right] {$+x$};

        \draw[axis] (3,2) -- (7,0) circle (2pt);
        \begin{scope}[xshift=3cm]
            % \lineann[0.5]{90}{1.3}{$h$};
            \lineann[1]{90}{2}{$R$};
        \end{scope}
        \begin{scope}[xshift=3cm, yshift=2cm]
            \lineann[1]{-28}{4.5}{$r$};
        \end{scope}
        \begin{scope}[xshift=3cm, yshift=0cm]
            \lineann[-1]{0}{4}{$x$};
        \end{scope}
    \end{tikzpicture} 
\end{figure}

Так как все заряды на кольце расположены на одном и том же расстоянии ($r=\sqrt{x^2+R^2}$) от произвольно выбранной точки наблюдения на оси кольца, то потенциал (при условии, что $\phi(x\to\infty)=0$) можно записать как

\begin{equation}
    \phi(x)=\int\limits_{(q)} k\frac{dq}{r}=k\frac{q}{r}
\end{equation}

Так как поле потенциально, то работа при перемещении не зависит от траектории, и по определению  потенциала такая работа будет

\begin{equation}
    A_{0\to l}=q'(\phi(0)-\phi(l))=kqq'(\frac{1}{R}-\frac{1}{\sqrt{R^2+l^2}})
\end{equation}
\begin{equation}
    A=10\cdot3\cdot10^3\cdot5\cdot3\cdot10^3(\frac{1}{25}-\frac{1}{\sqrt{25^2+50^2}})=9.95\cdot10^6 \text{ erg}
\end{equation}

\end{document}
