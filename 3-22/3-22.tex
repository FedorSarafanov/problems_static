\documentclass[a5paper,10pt]{article}
\def\source{/home/osabio/tex/templates}
\input{\source/head.tex}
\irodovE{3.22}
\usepackage{esint}
\begin{document}

\begin{tikzpict}
	\draw (0,0) circle (2cm);
	% \filldraw[even odd rule,inner co

	or=black!40,outer color=black!0] (0,0) circle (2);
	\draw[dashed, magenta] (0,0) circle (2.2cm);
	\draw[dashed, magenta] (0,0) circle (1.8cm);
	\foreach \s in {0,20,...,360}
	{
		\draw[->, blue!50] (0,0) ++ (\s:1) -- (\s:2.5);
		\draw[->, blue!50] (0,0) ++ (\s+10:1.5) -- (\s+10:3);
	}
	\draw[axis,->, black] (0,0) -- (3,0) node[right] {$+r$};
	% \lineann[1]{0}{2}{$2a$}
	% \begin{scope}[xshift=1cm]
	% 	\lineannn[-1]{0}{0.3}{$dx$}		
	% \end{scope}
	% \draw (0,-0.15) rectangle ++(2,0.3);
	% \draw[fill=magenta] (1,-0.15) rectangle ++(0.3,0.3);
	% % \draw (1.15,-0.1) node[below] {$dq$};
	% % \draw (-4,0) circle (0cm);
	% \vbLabel{0}{-0.1}{0};
	% \draw[axis,->] (0,0) -- ++(4,0) node[right] {$+x$};

	% \draw[fill=magenta] (3,0) coordinate (Y) circle (2pt);
	% \draw[force,->] (Y) -- ++ (0.5,0) node[below] {$d\vec{E}$};
\end{tikzpict}
Объемная плотность заряда
\begin{equation}
	\rho(r)=\rho_0(1-\frac{r}{R})
\end{equation}
Теорема Остроградского-Гаусса:
\begin{equation}
	\label{eq:og}
	\oiint\limits_{(S)}\vec{E}d\vec{S}=
	k\cdot 4\pi q_{in}
\end{equation}
где
\begin{equation}
	\label{eq:q}
	q_{in}=
	\iiint\limits_{(V)}\rho dV=
	\left\{
	\begin{aligned}
		\int\limits_0^r \rho_0(1-\frac{r}{R}) 4\pi r^2 dr=&
		\pi \rho_0\frac{r^3(4R-3r)}{3R},& \quad if \quad r<R\\
		\int\limits_0^R \rho_0(1-\frac{r}{R}) 4\pi r^2 dr=&
		\frac{1}{3}\pi R^3\rho_0,& \quad if \quad r\geq R
	\end{aligned}
	\right.
\end{equation}
С другой стороны,
\begin{equation}
	\label{eq:int}
	\oiint\limits_{(S)}\vec{E}d\vec{S}
		\overset{\vec{E}\parallel \vec{n}}{=}
	\oiint\limits_{(S)}E_rdS
		\overset{E(r)=const}{=}
	E\oiint\limits_{(S)}dS=ES=E\cdot4\pi r^2
\end{equation}
Из формул (\ref{eq:og}), (\ref{eq:q}), (\ref{eq:int}) следует
\begin{equation}
	E(r)=\frac{k}{r^2}q_{in}=
	\left\{
	\begin{aligned}
		k\pi& \rho_0\frac{r(4R-3r)}{3R},& \quad if \quad r<R\\
		k\pi& \rho_0\frac{ R^3}{3r^2},& \quad if \quad r\geq R
	\end{aligned}
	\right.	
\end{equation}
Найдем на оси $+r$ точку максимума напряженности:
\begin{equation}
	\frac{dE}{dr}=2 \pi  k \rho_0\frac{ (2 R-3 r)}{\ldots}=0
\end{equation}
Откуда точка максимума $r_{max}=\frac23R$

\begin{figure}[h]
	\centering
	\begin{tikzpicture}
		\begin{axis}[
			enlargelimits=false,
			ymax = 0.5,
			xtick=\empty,
			ytick=\empty,
 			axis x line*=bottom,
  			axis y line*=left, 	
  			xlabel=$r$,
  			ylabel=$E(r)$,
  			axis y line=middle,
			axis x line=middle,
			every axis x label/.style={
			    at={(ticklabel* cs:1.05)},
			    anchor=west,
			},
			every axis y label/.style={
			    at={(ticklabel* cs:1.05)},
			    anchor=south,
			},  			
		]
			\addplot[domain=0:1, samples=100, magenta]{x*(4-3*x)/3};
			\addplot[domain=1:2, samples=100, magenta]{1/3/x^2};

		\end{axis}
			\draw[dashed] (-0.1,0.385*13.15) node [left] {$E_{max}$} -- ++(3.5,0);
			\draw[dashed] (-0.1,0.385*9.9) node [left] {$E(R)$} -- ++(3.5,0);
			\draw[dashed] (2.3,-0.1) node [below] {$\frac23R$} -- ++(0,5.2);
			\draw[dashed] (3.4,-0.1) node [below] {$R$} -- ++(0,5.2);
	\end{tikzpicture}
	\caption{График функции $E(z)$}
	\label{fig:figure1}
\end{figure}
Соответственно максимальное значение напряженности будет
\begin{equation}
	E_{max}=
	E(\frac23R)=
	k\pi \rho_0\frac{2R(4R-2R)}{9R}=
	k\pi \rho_0\frac{4R}{9}
\end{equation}
\end{document}