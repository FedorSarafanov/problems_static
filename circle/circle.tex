\documentclass[a4paper,14pt]{extarticle}
\def\source{/home/osabio/tex/templates}
\input{\source/head.tex}
\wrote{Кольца}{Электростатика}
% Стыдить лжеца, шутить над дураком
% И спорить с женщиной — всё то же,
% Что черпать воду решетом:
% От сих троих избавь нас, боже!..

% М. Ю. Лермонтов.
\begin{document}

\begin{figure}[H]
    \centering
    \begin{tikzpicture}
        % \draw (0,0) node[left]{$\phi_0$} circle (1cm)
        %     circle (2cm);
        % \draw[fill=black] (0,0) circle (2pt);
        % \draw[axis,->] (0,0) -- (3,0) node[right] {$+r$};
        % %
        % \lineann[4]{90}{2}{$R_2$};
        % \lineann[3]{90}{1}{$R_1$};
        % \fill[magenta!10] (0,0) circle (2);
        % \fill[white] (0,0) circle (1.75);
        % \draw[black,thick] (0,0) circle (2);
        % \foreach \i in {1.75,2}{
        %     \draw[black!50] (0,0) circle (\i);
        % };
        % \foreach \angle in {0,10,...,360}{
        %     \draw[black,thick] (0,0) ++ (\angle:1.75) -- ++ (\angle:0.25);
        % };

        \draw[thick] (0,0) ++ (0,-2) node[below, blue] {$+q$} rectangle ++(0.1,4) ;
        \draw[thick] (8,0) ++ (0,-2) node[below, red] {$-q$} rectangle ++(-0.1,4);
        \draw[dashed, blue,->] (0,0) -- (9,0) node[thick,right] {$+x$};

        \draw[axis] (0,2) -- (4,0) circle (2pt);
        \lineann[1]{90}{2}{$R$};
        \begin{scope}[xshift=0cm, yshift=2cm]
            \lineann[1]{-28}{4.5}{$r$};
        \end{scope}
            \lineann[-1]{0}{8}{$l$};
    \end{tikzpicture} 
\end{figure}

Из геометрических соображений, расстояние от точки наблюдения с координатой $x$ до положительного кольца, где $a=l/2$:
\begin{equation}
	L_+=|x+a| \quad\Rightarrow\quad
	r_+=\sqrt{(x+a)^2+R^2}
\end{equation}
До отрицательного:
\begin{equation}
	L_-=|x-a|	\quad\Rightarrow\quad
	r_-=\sqrt{(x-a)^2+R^2}
\end{equation}

Потенциал кольца в точке на оси, расстояние от которой до любой точки кольца равно $r$:
\begin{equation}
	\phi=k\int\limits_{q} \frac{dq}{r}=\frac{kq}{r}
\end{equation}
Тогда суммарный потенциал обоих колец на оси
\begin{align}
	\phi(x)=\phi_++\phi_-
	=
	kq
	\left[	
	\frac{1}{
		\left(
			\left(x+a\right)^2+R^2
		\right)^{1/2}
	}
	-
	\frac{1}{
		\left(
			\left(x-a\right)^2+R^2
		\right)^{1/2}
	}	
	\right]
\end{align}
% Разложение дробей в ряд Лорана до второго члена $(x\to\infty, x\gg R\gg a)$ даст
% \begin{align}
% 	\phi=kq
% 	\left[	
% 		\frac{1}{x}-\frac{a}{x^2}-\frac{1}{x}-\frac{a}{x^2}
% 	\right]=-k\frac{2qa}{x^2}=-k\frac{p}{x^2}
% 	% =
% 	% \frac{kq}{R}
% 	% \left[	
% 	% \frac{\sqrt{\left(\frac{x-a}{R}\right)^2+1}-\sqrt{\left(\frac{x+a}{R}\right)^2+1}}{
% 	% 	\sqrt{
% 	% 	\left(
% 	% 		\left(\frac{x+a}{R}\right)^2+1
% 	% 	\right)
% 	% 	\left(
% 	% 		\left(\frac{x-a}{R}\right)^2+1
% 	% 	\right)		
% 	% 	}
% 	% }
% 	% \right]	
% 	% \left[
% 	% 	\left[
% 	% 		(x+a)^2+R^2
% 	% 	\right]^{-1/2}
% 	% 	-
% 	% 	\left[
% 	% 		(x-a)^2+R^2
% 	% 	\right]^{-1/2}		
% 	% \right]
% \end{align}
% Где
% \begin{equation}
% 	\sqrt{\left(\frac{x-a}{R}\right)^2+1}=1+\frac12\left(\frac{x-a}{R}\right)^2
% \end{equation}
% \begin{equation}
% 	\sqrt{\left(\frac{x+a}{R}\right)^2+1}=1+\frac12\left(\frac{x+a}{R}\right)^2
% \end{equation}
% И тогда
% \begin{equation}
% 	\phi(x)=
% 	\frac{kq}{R}
% 	\left[	
% 	\frac{
% 		1+\frac12\left(\frac{x-a}{R}\right)^2-
% 		1-\frac12\left(\frac{x+a}{R}\right)^2
% 	}{
% 		\left(
% 			1+\frac12\left(\frac{x-a}{R}\right)^2
% 		\right)
% 		\left(
% 			1+\frac12\left(\frac{x+a}{R}\right)^2
% 		\right)		
% 	}
% 	\right]=
% \end{equation}
% \begin{equation}
% 	=
% 	\frac{1kqR}{2}
% 	\left[	
% 	\frac{
% 		% \left(\frac{x-a}{R}\right)^2-
% 		% \left(\frac{x+a}{R}\right)^2
% 		-2a\cdot 2x
% 	}{
% 		a^4+4 a^2 R^2-2 a^2 x^2+4 R^4+4 R^2 x^2+x^4	
% 	}
% 	\right]=
% \end{equation}
Тогда
\begin{equation}
	E_x=-\diffp{\phi}{x}
	=
	kq
	\left[	
	\frac{x+a}{
		\left(
			\left(x+a\right)^2+R^2
		\right)^{3/2}
	}
	-
	\frac{x-a}{
		\left(
			\left(x-a\right)^2+R^2
		\right)^{3/2}
	}	
	\right]
\end{equation}

Это точное рассмотрение, которое верно вне зависимости от размера элементов системы и положения точки наблюдения на оси.

При рассмотрении на бесконечности будет проще решить задачу, используя выведенное ранее выражение для потенциала диполя, предпологая, что два кольца образуют систему из диполей, образованных зарядом $dq$ на одном кольце и $-dq$ на другом. 

Очевидно, что в силу принципа суперпозиции и радиальной симметрии задачи задачу можно решать для диполя ${+q,-q}$ следующего вида:

\begin{figure}[H]
    \centering
    \begin{tikzpicture}
        \draw[thick] (0,2) circle (2pt) node[above, blue] {$+q$};% rectangle (0.1,4) ;
        \draw[thick] (8,2) circle (2pt) node[above, red] {$-q$};% rectangle ++(-0.1,4);
        \draw[<-, thick, blue] (0,2) -- node[above] {$\vec{l}$} (8,2);
        \draw[dashed, blue,->] (0,0) -- (9,0) node[thick,right] {$+x$};

        \draw[axis] (4,2) -- (8,0) circle (2pt);
        \lineann[1]{90}{2}{$R$};
        \begin{scope}[xshift=4cm, yshift=2cm]
            \lineann[1]{-28}{4.5}{$r$};
        \end{scope}
            % \lineann[-1]{0}{8}{$l$};
    \end{tikzpicture} 
\end{figure}

Тогда на достаточно больших расстояниях ($r\gg l $)
\begin{equation}
	\phi(\vec{r})=k\frac{q(\vec{l},\,\vec{r})}{r^3}
\end{equation}

Очевидно, что при наложении условия $r \gg R$ выполняется $r\to x$, и тогда 
\begin{equation}
	\phi(x)=k\frac{(\vec{p},x\vec{i})}{x^3}=-sign(x)k\frac{p}{x^2}
\end{equation}

% Где на бесконечности $\cos(l,r)=1$, и 
% \begin{equation}
% 	\frac{|x|}{x}=\left\{
% 	\begin{aligned}
% 		-1,\quad x>0\\
% 		1,\quad x<0
% 	\end{aligned}=-sign(x)
% 	\right.
% \end{equation}

% % \begin{equation}
	
% % \end{equation}

% При $x\gg R\gg a$
% \begin{equation}
% 	E_x=
% 	kq
% 	\left[	
% 	\frac{1}{
% 		\left(
% 			x-a
% 		\right)^2
% 	}
% 	-
% 	\frac{1}{
% 		\left(
% 			x+a
% 		\right)^2
% 	}
% 	\right]
% % \end{equation}
% \begin{equation}
% 	\phi(x)=
% 	\frac{kq}{x}
% 	\left[	
% 	\frac{1}{
% 		1+a/x
% 	}
% 	-
% 	\frac{1}{
% 		1-a/x
% 	}
% 	\right]=-kq\frac{2a}{x^2-a^2}
% \end{equation}
% \begin{equation}
% 	E_x=%-\diffp{\phi}{x}
% 	% =
% 	\frac{kq}{x^2}
% 	\left[	
% 	\frac{1}{
% 		(1+a/x)^2
% 	}
% 	-
% 	\frac{1}{
% 		(1-a/x)^2
% 	}
% 	\right]	
% \end{equation}
% При наложении дополнительного условия $x\gg a$ разложим дроби в ряд Тейлора до второго члена:
% \begin{equation}
% 	\frac{1}{1\pm a/x}=(1\pm a/x)^{-1}=1\mp a/x+\ldots \approx 1\mp a/x
% \end{equation}
% \begin{equation}
% 	\frac{1}{(1\pm a/x)^2}=(1\pm a/x)^{-2}=1\mp 2a/x+\ldots \approx 1\mp 2a/x
% \end{equation}
% Тогда
% \begin{equation}
% 	\phi(x)=
% 	\frac{kq}{x}
% 	\left[	
% 	1-a/x-(1+a/x)
% 	\right]
% 	=
% 	-2a\frac{kq}{x^2}=-k\frac{p}{x^2}
% \end{equation}
Тогда найдем $E_x(x)$:
\begin{equation}
	E_x=-\diffp{\phi}{x}=-k\frac{2p}{x^3}
\end{equation}


\begin{figure}[h]
\begin{minipage}[h]{0.49\linewidth}
	\centering
	\begin{tikzpicture}
		\begin{axis}[
			enlargelimits=false,
			% ymax = 0.5,
			xtick=\empty,
			ytick=\empty,
 			axis x line*=bottom,
  			axis y line*=left, 	
  			xlabel=$x$,
  			ylabel=$\phi(x)$,
  			axis y line=middle,
			axis x line=middle,
			every axis x label/.style={
			    at={(ticklabel* cs:1.05)},
			    anchor=west,
			},
			every axis y label/.style={
			    at={(ticklabel* cs:1.05)},
			    anchor=south,
			},  			
		]
			% \addplot[domain=-10:10, samples=500]{(x-1)/((x-1)^2+1)^(3/2)-(x+1)/((x+1)^2+1)^(3/2)};

			\addplot[domain=-10:10, samples=100]{%
			1/sqrt((x+1)^2+1^2)-
			1/sqrt((x-1)^2+1^2)%
			};		

			% \addplot[domain=-10:10, samples=100]{%
			% (abs(x-1)-abs(x+1))/(x^2+1)
			% };				

		\end{axis}
	\end{tikzpicture}
\end{minipage}
\hfill
\begin{minipage}[h]{0.49\linewidth}
	\centering
	\begin{tikzpicture}
		\begin{axis}[
			enlargelimits=false,
			% ymax = 0.5,
			xtick=\empty,
			ytick=\empty,
 			axis x line*=bottom,
  			axis y line*=left, 	
  			xlabel=$x$,
  			ylabel=$E_x(x)$,
  			axis y line=middle,
			axis x line=middle,
			every axis x label/.style={
			    at={(ticklabel* cs:1.05)},
			    anchor=west,
			},
			every axis y label/.style={
			    at={(ticklabel* cs:1.05)},
			    anchor=south,
			},  			
		]
			% \addplot[domain=-10:10, samples=500]{(x-1)/((x-1)^2+1)^(3/2)-(x+1)/((x+1)^2+1)^(3/2)};

			% \addplot[domain=-10:10, samples=500]{(x-1)/((x-1)^2)^(3/2)-(x+1)/((x+1)^2)^(3/2)};			
			% \addplot[domain=-10:10, samples=100]{%
			% (x+1)/((x+1)^2+1^2)^(3/2)-
			% (x-1)/((x-1)^2+1^2)^(3/2)
			% };			
			\addplot[domain=-10:10, samples=180]{%
			(x+1)/((x+1)^2+1^2)^(3/2)-
			(x-1)/((x-1)^2+1^2)^(3/2)
			};	

			% \addplot[domain=-10:10, samples=180]{%
			% (x)/(x^2+0.001*2*x+1^2)^(3/2)-
			% (x)/(x^2-0.001*2*x+1^2)^(3/2)
			% };
			% \addplot[domain=-1:1, samples=100]{%
			% 1/((x+0.1)^2)
			% };										
		\end{axis}
	\end{tikzpicture}≫
\end{minipage}
\end{figure}
%
\end{document}