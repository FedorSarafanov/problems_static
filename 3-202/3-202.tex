\documentclass[a4paper,14pt]{extarticle}
\def\source{/home/osabio/tex/templates}
%   \begin{vmatrix}
%   \text{душ}&  \text{душа} &  \text{душе}\\
%   \text{душой}&  \text{душем} &  \text{души}\\
%   \text{душа}&  \text{душем} &  \text{душесса}\\
% \end{vmatrix}
\input{\source/head.tex}
\irodov{3.202}{Электростатика}
\usepackage{mathrsfs}
\tikzset{
  pics/carc/.style args={#1:#2:#3}{
    code={
      \draw[pic actions] (#1:#3) arc(#1:#2:#3);
    }
  }
}
\begin{document}
% \begin{equation}
% \end{equation}
Емкость конденсатора без пластинки
\begin{equation}
  C=\frac{S}{4\pi k}\cdot\frac{1}{d}
\end{equation}
Нетрудно показать, что емкость конденсатора с пластинкой
\begin{equation}
  C'=\frac{S}{4\pi k}\cdot\frac{1}{d-\eta d}
\end{equation}
Тогда
\begin{equation}
  \Delta W = W_2-W_1=\frac{CU^2}{2}-\frac{C'U^2}{2}=
  -\frac{SU^2}{8\pi k}
  \left[
    \frac{1}{d-\eta d}-
    \frac{1}{d}
  \right]=-\frac{SU^2}{8\pi k }\cdot\frac{\eta d}{d^2(1-\eta)}
\end{equation}
\begin{equation}
  \Delta W = -\frac{CU^2}{2} \frac{\eta}{1-\eta}=-\frac32\cdot\frac{CU^2}{2}=-0.15 \text{ мДж}
\end{equation}
Источник при этом совершит работу
\begin{equation}
  A_{s}=U\Delta q=U(UC'-UC)=U^2(C'-C)=-2\Delta W
\end{equation}
Откуда
\begin{equation}
  A_{s}=\frac32 CU^2
\end{equation}
Отсюда найдем механическую работу $A_m$:
\begin{equation}
  A_m=\Delta W + A_s = \frac32\frac{CU^2}{2}=0.15 \text{ мДж}
\end{equation}


\end{document}