\documentclass[a5paper,10pt]{article}
\def\source{/home/osabio/tex/templates}
\input{\source/head.tex}
\irodovE{3.12}

\begin{document}

\begin{tikzpict}
	\draw (0,0) circle (1cm);
	\draw (-4,0) circle (0cm);
	\draw[axis,->] (0,0) -- ++(0,1.5) node[above] {$+y$};
	\draw[axis,->] (0,0) -- ++(1.5,0) node[right] {$+x$};

	\draw[axis] (0,0) -- ++ (45:1) coordinate (X);

	\draw[force,->] (0,0) -- ++ (135+90:0.4) node[below] {$d\vec{E}$};
	\draw[fill=magenta] (X) circle (2pt) node [right, xshift=0.5em] %
		{$dq(\phi)=dl\cdot \lambda_0 \cos\phi$};
	\draw[fill=magenta] (0,0) circle (2pt);% node [left]		{$P$};
\end{tikzpict}
Рассмотрим напряженность в центре кольца.
\begin{equation}
	d\vec{E}=
		dE_x\cdot \vec{i}+
		dE_y\cdot \vec{j}\\	
\end{equation}
Напряженность элементарного (точечного) заряда $dq$
\begin{equation}
	dE=\frac{k\cdot dq(\phi)}{R^2}
\end{equation}
Запишем дифференциал напряженности в проекциях:
\begin{gather}
	dE_x=\frac{k\cdot dq(\phi)}{R^2}\cos\phi\\
	dE_y=\frac{k\cdot dq(\phi)}{R^2}\sin\phi
\end{gather}
Для перевода интегрирования в сферические координаты, выразим диффиренциал дуги через дифференциал азимутального угла:
\begin{equation}
	\phi=\frac{l}{R}
	\quad\Rightarrow\quad	
	dl=R\cdot d\phi
\end{equation}
\begin{gather}
	E_x=\frac{k\lambda_0}{R^2}\int\limits_{0}^{L}   	\cos^2\phi \cdot dl=
		\frac{k\lambda_0}{R^2}\int\limits_{0}^{2\pi}   	\cos^2\phi \cdot R\cdot d\phi=\frac{k\lambda_0\pi}{R}\\
	E_y=\frac{k\lambda_0}{R^2}\int\limits_{0}^{L}   	\cos\phi\sin\phi \cdot dl=
		\frac{k\lambda_0}{R^2}\int\limits_{0}^{2\pi}   	\cos\phi\sin\phi \cdot R\cdot d\phi=0\\		
	E_y=\frac{k\lambda_0}{R^2}R\int\limits_{0}^{2\pi}   	\cos\phi\sin\phi \cdot d\phi=0
\end{gather}
Тогда полная напряженность в центре кольца
\begin{equation}
	E=k\frac{\pi\lambda_0}{R}
\end{equation}
\newpage
\begin{tikzpict}
	\draw (0,-1) -- (0,1);
	\draw (-4,0) circle (0cm);
	\draw[axis,->] (0,0) -- ++(0,1.5) node[above] {$+y$};
	\draw[axis,->] (0,0) -- ++(4,0) node[right] {$+z$};

	\draw[] (0,0) -- ++ (90:1) coordinate (X);

	\draw[fill=magenta] (X) circle (2pt) node [right, xshift=0.5em, yshift=0.5em] %
		{$dq(\phi)=dl\cdot \lambda_0 \cos\phi$};
	\draw[fill=magenta] (3,0) coordinate (Y) circle (2pt);% node [left]		{$P$};
	\draw[axis] (X) -- (Y);
	\draw[force,->] (Y) -- ++ (1.5/2,-0.5/2) node[below] {$d\vec{E}$};
\end{tikzpict}
% Рассмотрим напряженность в центре кольца.
	% dE_y=\frac{k\cdot dq(\phi)}{r^2}\sin\phi
% \begin{equation}
% 	d\vec{E}=
% 		dE_x\cdot \vec{i}+
% 		dE_y\cdot \vec{j}\\	
% \end{equation}
Напряженность элементарного (точечного) заряда $dq$
\begin{equation}
	dE=\frac{k\cdot dq(\phi)}{r^2}
\end{equation}
Запишем дифференциал напряженности в проекциях:
\begin{gather}
	dE_x=\frac{k\cdot dq(\phi)}{r^2}\cos\phi\sin\Theta\\
	dE_y=\frac{k\cdot dq(\phi)}{r^2}\sin\phi\sin\Theta\\
	dE_z=\frac{k\cdot dq(\phi)}{r^2}\cos\Theta
\end{gather}
Где
\begin{equation}
	\sin\Theta=\frac{R}{r}=\frac{R}{\sqrt{R^2+z^2}}\quad, \quad
	\cos\Theta=\frac{z}{r}=\frac{z}{\sqrt{R^2+z^2}}
\end{equation}
Для перевода интегрирования в сферические координаты, выразим диффиренциал дуги через дифференциал азимутального угла:
\begin{equation}
	\phi=\frac{l}{R}
	\quad\Rightarrow\quad	
	dl=R\cdot d\phi
\end{equation}
\begin{gather}
	E_x=\sin\Theta
		\frac{k\lambda_0}{r^2}
		\int\limits_{0}^{L}   	
		\cos^2\phi \cdot dl=
		\sin\Theta%
		\frac{k\lambda_0}{r^2}
		\int\limits_{0}^{2\pi}
		\cos^2\phi \cdot R\cdot d\phi=
		\frac{k\lambda_0\pi R^2}{r^3}\\
	E_y=\sin\Theta
		\frac{k\lambda_0}{r^2}
		\int\limits_{0}^{L}   	
		\cos\phi\sin\phi \cdot dl=
		\sin\Theta%
		\frac{k\lambda_0}{r^2}
		\int\limits_{0}^{2\pi}   	
		\cos\phi\sin\phi \cdot R\cdot d\phi=
		0\\		
	E_z=\cos\Theta
		\frac{k\lambda_0}{R^2}
		\int\limits_{0}^{L}   	
		\cos\phi\cdot dl=
		\cos\Theta%
		\frac{k\lambda_0}{R^2}
		\int\limits_{0}^{2\pi}   	
		\cos\phi\cdot R\cdot d\phi=
		0
\end{gather}
Тогда полная напряженность в точке наблюдения
\begin{equation}
	E=
	k\frac{\pi\lambda_0R^2}{r^3}=
	k\frac{\pi\lambda_0R^2}{(R^2+z^2)^{3/2}}
\end{equation}
При $z \gg R$
\begin{equation}
	E=
	k\frac{\pi\lambda_0R^2}{(z^2)^{3/2}}=
	k\frac{\pi\lambda_0R^2}{z^3}
\end{equation}
\end{document}