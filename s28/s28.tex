\documentclass[a4paper,14pt]{extarticle}
\def\source{/home/osabio/tex/templates}
\input{\source/head.tex}
\yakovlev{28}{Электростатика}
% Стыдить лжеца, шутить над дураком
% И спорить с женщиной — всё то же,
% Что черпать воду решетом:
% От сих троих избавь нас, боже!..

% М. Ю. Лермонтов.
\begin{document}

\begin{figure}[H]
    \centering
    \begin{tikzpicture}

        % \draw[thick] (0,0) ++ (0,-2) node[below, blue] {$+q$} rectangle ++(0.1,4) ;
        % \draw[thick] (8,0) ++ (0,-2) node[below, red] {$-q$} rectangle ++(-0.1,4);
        \draw[dashed, blue,->] (-3,0) -- (9,0) node[thick,right] {$+x$};
        \draw[draw=none] (0,4) (0,-4);
        \fill[red] (0,0) circle (2pt) node[above,red] {$+q$};
        \fill[red] (4,0) circle (4pt) node[above,red] {$+4q$};
        \draw[magenta] (4+0.48*4,0) node[red] {$\times$};
        \draw[magenta] (-0.18*4,0) node[red] {$\times$};
        \draw[magenta] (4/3,0) node[red] {$\times$};
        % \draw[axis] (0,2) -- (4,0) circle (2pt);
        % \lineann[1]{90}{2}{$R$};
        % \begin{scope}[xshift=0cm, yshift=2cm]
        %     \lineann[1]{-28}{4.5}{$r$};
        % \end{scope}
           \lineann[-1]{0}{4}{$d$};
    \end{tikzpicture} 
\end{figure}

Поле, создаваемое системой точечных зарядов:
\begin{equation}
	\vec{E}={k\cdot q}
	\left(
		\frac{\hat{r_1}}{r_1^2}+
		\frac{4\hat{r_2}}{r_2^2}
	\right)
\end{equation}

Из геометрических соображений, точка с нулевой напряженностью должна находится между зарядами, и решение такого уравнения
\begin{equation}
	0=
	% \left(
		\frac{1}{x_0^2}+
		\frac{4}{(d-x_0)^2} 
	% \right)
\end{equation}
имеющее физический смысл:
\begin{equation}
	x_0=\frac{d}{3}
\end{equation}
Потенциал в этой точке
\begin{equation}
	\phi(x_0)=\frac{kq}{x_0}+\frac{4kq}{d-x_0}=\frac{9kq}{d}
\end{equation}
Найдем точки на оси $x$ с таким же потенциалом:
\begin{equation}
	\left\{
	\begin{aligned}
		\frac{kq}{l}+\frac{4kq}{d+l}=\frac{9kq}{d}&, \quad x>d& \quad\Rightarrow\quad l=&\frac29d(\sqrt{10}-1)\approx0.48d\\
		\frac{kq}{d+l}+\frac{4kq}{d}=\frac{9kq}{d}&, \quad x<0& \quad\Rightarrow\quad l=&\frac{\sqrt{13}-2}{9}d\approx0.18d
	\end{aligned}
	\right.
\end{equation}


%
\end{document}