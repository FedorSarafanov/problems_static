\documentclass[a5paper,10pt]{article}\usepackage[usenames,dvipsnames]{color}\usepackage{extsizes,cmap,graphicx,misccorr,indentfirst,makecell,multirow,ulem,geometry,amssymb,amsfonts,amsmath,amsthm,titlesec,float,fancyhdr,wrapfig,tikz}\usepackage[T2A]{fontenc}\usepackage[utf8x]{inputenc}\usepackage[english, russian]{babel}\usetikzlibrary{decorations.pathreplacing,decorations.pathmorphing,patterns,calc,scopes,arrows,through,positioning,shapes.misc}\graphicspath{{img/}}\linespread{1.3}\frenchspacing\geometry{left=1cm, right=1cm, top=2cm, bottom=1cm, bindingoffset=0cm}\pagestyle{fancy}\fancyhead{}\fancyhead[R]{Сарафанов Ф.Г.} 
\fancyhead[C]{Электростатика}
\fancyhead[L]{Сивухин - №84} 
\fancyfoot{}
\renewcommand{\labelenumii}{\theenumii)}
\tikzset{
	force/.style={>=latex,draw=blue,fill=blue,>=triangle 45},
    axis/.style={densely dashed,blue},
    interface1/.style={draw=gray!60,.
        postaction={draw=gray!60,decorate,decoration={border,angle=-135,
        amplitude=0.3cm,segment length=2mm}}},
    interface/.style={
        pattern = north east lines,
        draw    = none,
        pattern color=gray!60,          
    },
    plank/.style={
        fill=black!60, 
        draw=black,
        minimum width=3cm,
        inner ysep=0.1cm,
        outer sep=0pt,
        yshift=0.75cm,
        pattern = north east lines,
        pattern color=gray!60, 
    },
    cargo/.style={
        rectangle,
        fill=black!70,              
        inner sep=2.5mm,
    }	
}

% Draw line annotation
% Input:
%   #1 Line offset (optional)
%   #2 Line angle
%   #3 Line length
%   #5 Line label
% Example:
%   \lineann[1]{30}{2}{$L_1$}
\newcommand{\lineann}[4][0.5]{%
    \begin{scope}[rotate=#2, blue,inner sep=2pt]
        \draw[dashed, blue!40] (0,0) -- +(0,#1)
            node [coordinate, near end] (a) {};
        \draw[dashed, blue!40] (#3,0) -- +(0,#1)
            node [coordinate, near end] (b) {};
        \draw[|<->|] (a) -- node[fill=white] {#4} (b);
    \end{scope}
}
\begin{document}

\begin{figure}[H]
    \centering
    \begin{tikzpicture}
        \draw (0,0) node[left]{$\phi_0$} circle (1cm)
            circle (2cm);
        \draw[fill=black] (0,0) circle (2pt);
        \draw[axis,->] (0,0) -- (3,0) node[right] {$+r$};
        %
        \lineann[4]{90}{2}{$R_2$};
        \lineann[3]{90}{1}{$R_1$};
    \end{tikzpicture} 
\end{figure}
Будем считать заранее выведенным поле $E(r)$ равномерно заряженной сферы произвольного радиуса $R$:
\begin{equation}
    E_R(r)=\left\{
    \begin{aligned}
        0, \quad r<R\\
        k\frac{q}{r^2}=k\frac{\sigma\cdot4\pi R^2}{r^2}, \quad r\geq R
    \end{aligned}
    \right.
\end{equation}
Найдем распределение потенциала  (при условии $\phi(r\to\infty)=0$):
\begin{equation}
    \phi_R(r)-\phi_R(r\to\infty)=\phi_R(r)=
    \int\limits_r^\infty k\frac{q}{r^2}=k\frac{q}{r}, \quad r\ge R
\end{equation}
\begin{equation}
    \phi_R(R)=k\frac{q}{R}, \quad \phi_R(0)-\phi_R(R)=\int\limits_0^R 0 dr=0 \Rightarrow \phi_R(0)=k\frac{q}{R}
\end{equation}
где $q=q(R)=\sigma\cdot 4\pi R^2$.

Тогда по принципу суперпозиции потенциал в центре шара будет
\begin{equation}
    \phi_0=\phi_{R_1}(0)+\phi_{R_2}(0)=k\sigma\cdot4\pi (R_1+R_2)
\end{equation}
С учетом того, что в СГСЭ $k=1$, а 300 В = 1, 
\begin{equation}
    \sigma=\frac{\phi_0}{k\cdot4\pi (R_1+R_2)}=\frac{1}{4\pi\cdot 30}\approx0.00265\ SGSE
\end{equation}

\end{document}
