\documentclass[a5paper,10pt]{article}\usepackage[usenames,dvipsnames]{color}\usepackage{extsizes,cmap,graphicx,misccorr,indentfirst,makecell,multirow,ulem,geometry,amssymb,amsfonts,amsmath,amsthm,titlesec,float,fancyhdr,wrapfig,tikz,pgfplots}\usepackage[T2A]{fontenc}\usepackage[utf8x]{inputenc}\usepackage[english, russian]{babel}\usetikzlibrary{decorations.pathreplacing,decorations.pathmorphing,patterns,calc,scopes,arrows,through,positioning,shapes.misc}\graphicspath{{img/}}\linespread{1.3}\frenchspacing\geometry{left=1cm, right=1cm, top=2cm, bottom=1cm, bindingoffset=0cm}\pagestyle{fancy}\fancyhead{}\fancyhead[R]{Сарафанов Ф.Г.} 
\fancyhead[C]{Электростатика}
\fancyhead[L]{Задача под запись -- <<Диск>>} 
\fancyfoot{}
\renewcommand{\labelenumii}{\theenumii)}
\tikzset{
	force/.style={>=latex,draw=blue,fill=blue,>=triangle 45},
    axis/.style={densely dashed,blue},
    interface1/.style={draw=gray!60,.
        postaction={draw=gray!60,decorate,decoration={border,angle=-135,
        amplitude=0.3cm,segment length=2mm}}},
    interface/.style={
        pattern = north east lines,
        draw    = none,
        pattern color=gray!60,          
    },
    plank/.style={
        fill=black!60, 
        draw=black,
        minimum width=3cm,
        inner ysep=0.1cm,
        outer sep=0pt,
        yshift=0.75cm,
        pattern = north east lines,
        pattern color=gray!60, 
    },
    cargo/.style={
        rectangle,
        fill=black!70,              
        inner sep=2.5mm,
    }	
}

% Draw line annotation
% Input:
%   #1 Line offset (optional)
%   #2 Line angle
%   #3 Line length
%   #5 Line label
% Example:
%   \lineann[1]{30}{2}{$L_1$}
\newcommand{\lineann}[4][0.5]{%
    \begin{scope}[rotate=#2, blue,inner sep=2pt]
        \draw[dashed, blue!40] (0,0) -- +(0,#1)
            node [coordinate, near end] (a) {};
        \draw[dashed, blue!40] (#3,0) -- +(0,#1)
            node [coordinate, near end] (b) {};
        \draw[|<->|] (a) -- node[fill=white] {#4} (b);
    \end{scope}
}
\begin{document}

\begin{figure}[H]
    \centering
    \begin{tikzpicture}
        % \draw (0,0) node[left]{$\phi_0$} circle (1cm)
        %     circle (2cm);
        % \draw[fill=black] (0,0) circle (2pt);
        % \draw[axis,->] (0,0) -- (3,0) node[right] {$+r$};
        % %
        % \lineann[4]{90}{2}{$R_2$};
        % \lineann[3]{90}{1}{$R_1$};
        \draw[black,thick] (0,0) circle (2);
        \fill[magenta!10] (0,0) circle (1.5);
        \fill[white] (0,0) circle (1.25);
        \foreach \i in {0,0.25,...,2}{
            \draw[black!50] (0,0) circle (\i);
        };
        \foreach \angle in {0,10,...,360}{
            \draw[black,thick] (0,0) ++ (\angle:1.25) -- ++ (\angle:0.25);
        };

        \draw[thick] (3,0) ++ (0,-2) rectangle ++(0.1,4);
        \draw[dashed, blue,->] (3,0) -- (8,0) node[thick,right] {$+x$};

        \draw[axis] (3,1.3) -- (7,0) circle (2pt);
        \begin{scope}[xshift=3cm]
            \lineann[0.5]{90}{1.3}{$h$};
            \lineann[1]{90}{2}{$R$};
        \end{scope}
        \begin{scope}[xshift=3cm, yshift=1.3cm]
            \lineann[1]{-18.9}{4.1}{$r$};
        \end{scope}
        \begin{scope}[xshift=3cm, yshift=0cm]
            \lineann[-1]{0}{4}{$x$};
        \end{scope}
    \end{tikzpicture} 
\end{figure}
Разобьем диск на тонкие кольца шириной $dh$, радиусом $h$. 

Такие кольца разобьем на площадки длиной $dl$, с зарядом $dq=\sigma\cdot dl\cdot dh$. 
Тогда найдем потенциал по принципу суперпозиции, считая, что для точечного заряда $\phi(r\to\infty)=0$ и его потенциал $\phi=k\cdot dq/r$:
\begin{equation}
    \phi(x)=\int\limits_0^R\int\limits_0^{2\pi h} k\sigma \frac{dh\cdot dl}{\sqrt{x^2+h^2}}=
    k\sigma2\pi 
    \int\limits_0^{R}\frac{hdh}{\sqrt{x^2+h^2}}=
\end{equation}
\begin{equation}
    =k\sigma\pi \int\limits_0^R \frac{d[x^2+h^2]}{\sqrt{x^2+h^2}}=2k\sigma\pi(x^2+h^2)^{1/2}\bigg|_0^R=2k\sigma\pi\left[\sqrt{x^2+R^2}-|{x}|\right]
\end{equation}

Зная распределение потенциала вдоль оси, можем найти $E_x$:
\begin{equation}
    E_x(x)=-\nabla \phi(x)=-\frac{\partial}{\partial x}2k\sigma\pi\left[\sqrt{x^2+R^2}-|{x}|\right]=
    \end{equation}
\begin{equation}
    =2k\sigma\pi\left[sign(x)-\frac{x}{\sqrt{x^2+R^2}}\right]
\end{equation}

% \begin{figure}[h]
%     \centering
%     \begin{tikzpicture}[baseline, scale=0.7]
%         \begin{axis}[
%            yticklabel pos=right,
%             axis x line = center,   % rien en y     
%             axis y line = center,     % sans axe y
%             xlabel={$x$},
%             ylabel={$\phi(x)$},
%             % xmax=2.5,
%             % xmin=-2.5,
%             ymax=1.1,
%             yticklabels={,,}
%             xticklabels={,,},
%             ticks=none,
%         ]
%             % density of Normal distribution:
%             \newcommand\MU{0}
%             \newcommand\SIGMA{1e-3}
%             \addplot[red,domain=-5:5,samples=200]
%                {sqrt(x^2+1)-abs(x)};
%         \end{axis}
%     \end{tikzpicture}
%     \caption{$\phi(x)$}
%     \label{fig:figure1}
% \end{figure}

\end{document}
