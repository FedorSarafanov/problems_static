\documentclass[a5paper,10pt]{article}\usepackage[usenames,dvipsnames]{color}\usepackage{extsizes,cmap,graphicx,misccorr,indentfirst,makecell,multirow,ulem,geometry,amssymb,amsfonts,amsmath,amsthm,titlesec,float,fancyhdr,wrapfig,tikz,pgfplots}\usepackage[T2A]{fontenc}\usepackage[utf8x]{inputenc}\usepackage[english, russian]{babel}\usetikzlibrary{decorations.pathreplacing,decorations.pathmorphing,patterns,calc,scopes,arrows,through,positioning,shapes.misc}\graphicspath{{img/}}\linespread{1.3}\frenchspacing\geometry{left=1cm, right=1cm, top=2cm, bottom=1cm, bindingoffset=0cm}\pagestyle{fancy}\fancyhead{}\fancyhead[R]{Сарафанов Ф.Г.} 
\fancyhead[C]{Осеннее падение}
\fancyhead[L]{Поэма} 
\fancyfoot{}
\renewcommand{\labelenumii}{\theenumii)}

\begin{document}

\begin{center}
   \begin{verse}
Однажды в ранне-осенний денёк\\
Ваня поехал смотреть москвичёк.\\
\smallskip
\smallskip

Чтобы на листьях не наебнуться,\\
В автобус ему хотелось вернуться.\\
\smallskip
\smallskip
Но не внимал он разума гласу,\\
Сел и поехал на веле до разу.\\
\smallskip
\smallskip
Листья упавшие ямы скрывали...\\
Ямы скрывшиеся вилку сломали.\\
\smallskip
\smallskip
Помяты крылья, помят пассажир - \\
Ваня чудом дтп пережил.\\
Но не прошло без легкого шока-\\
Тормоза ручку помял он до штока.\\
Две ямы попались ему на пути\\
Одну он запомнил, мать их ебти.\\


\end{verse} 
\end{center}

\end{document}
