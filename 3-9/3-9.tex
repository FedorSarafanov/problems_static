\documentclass[a5paper,10pt]{article}
\def\source{/home/osabio/tex/templates}
\input{\source/head.tex}
\irodovE{3.9}
% Наш паровоз летит, 
% Колёса гнутые, 
% А нам всё похую, 
% Мы ебанутые! 
\begin{document}
\begin{tikzpict}
	\draw (0,-1) -- (0,1);
	% \draw (-4,0) circle (0cm);
	\lineann[1]{90}{1}{$R$}
	\begin{scope}[yshift=1cm]
		\lineann[1.5]{-19}{3.16}{$r$}		
	\end{scope}	
	\draw[axis,->] (0,0) -- ++(0,1.5) node[above] {$+y$};
	\draw[axis,->] (0,0) -- ++(4,0) node[right] {$+z$};

	\draw[] (0,0) -- ++ (90:1) coordinate (X);

	\draw[fill=magenta] (X) circle (2pt) node [right, xshift=0.5em, yshift=0.5em] %
		{$dq(\phi)=dl\cdot \lambda$};
	\draw[fill=magenta] (3,0) coordinate (Y) circle (2pt);% node [left]		{$P$};
	\draw[axis] (X) -- (Y);
	\draw[force,->] (Y) -- ++ (1.5/2,-0.5/2) node[below] {$d\vec{E}$};
\end{tikzpict}
\begin{equation}
	\oiint\limits_{(S)} \overrightarrow{E} d\overrightarrow{S} = 4 \pi 
\end{equation}
Линейная плотность заряда на кольце
\begin{equation}
	\lambda=\frac{q}{2\pi R}
\end{equation}
% Рассмотрим напряженность в центре кольца.
	% dE_y=\frac{k\cdot dq(\phi)}{r^2}\sin\phi
% \begin{equation}
% 	d\vec{E}=
% 		dE_x\cdot \vec{i}+
% 		dE_y\cdot \vec{j}\\	
% \end{equation}
Напряженность элементарного (точечного) заряда $dq$
\begin{equation}
	dE=\frac{k\cdot \lambda dl}{r^2}
\end{equation}
С другой стороны, если перейти в полярную систему координат, то
\begin{equation}
	dl=Rd\phi
\end{equation}
И тогда
\begin{equation}
	dE=\frac{k\cdot \lambda Rd\phi}{r^2}=\frac{k\lambda R}{r^2} d\phi
\end{equation}
Запишем дифференциал напряженности в проекциях:
\begin{gather}
	dE_x=\frac{k\lambda R}{r^2} \cos\phi\sin\Theta\ d\phi, \quad
	E_x=\frac{k\lambda R}{r^2}\sin\Theta\int\limits_0^{2\pi}\cos\phi\ d\phi=0\\
	dE_y=\frac{k\lambda R}{r^2} \sin\phi\sin\Theta\ d\phi
	, \quad
	E_y=\frac{k\lambda R}{r^2}\sin\Theta\int\limits_0^{2\pi}\sin\phi\ d\phi=0\\
	dE_z=\frac{k\lambda R}{r^2} \cos\Theta\ d\phi
	, \quad
	E_z=\frac{k\lambda R}{r^2}\cos\Theta\int\limits_0^{2\pi}d\phi=\frac{k 2\pi\lambda Rx}{r^3}=\frac{kqx}{r^3}
\end{gather}
Где
\begin{equation}
	\sin\Theta=\frac{R}{r}=\frac{R}{\sqrt{R^2+z^2}}\quad, \quad
	\cos\Theta=\frac{z}{r}=\frac{z}{\sqrt{R^2+z^2}}
\end{equation}

Тогда полная напряженность в точке наблюдения
\begin{equation}
	E=E_z=
	k\frac{qz}{r^3}=k\frac{qz}{(R^2+z^2)^{3/2}}
\end{equation}
При $z \gg R$
\begin{equation}
	E=
	k\frac{q}{z^2}
\end{equation}
Найдем на оси $z$ точку максимума напряженности.
\begin{equation}
	\frac{dE}{dz}=\frac{kq(R^2-2z^2)}{\ldots}=0
\end{equation}
На расстоянии $z'$ (решение уравнения выше) $E$ будет максимальным:
\begin{equation}
	z'=\frac{R}{\sqrt{2}}
\end{equation}
\begin{equation}
	E_{max}=E(z')=k\frac{qR}{\sqrt{2}(R^2+R^2/2)^{3/2}}=
	\frac{2kq}{3\sqrt{3}R^2}
\end{equation}

\begin{figure}[h]
	\centering
	\begin{tikzpicture}
		\begin{axis}[
			enlargelimits=false,
			ymax = 0.5,
			xtick=\empty,
			ytick=\empty,
 			axis x line*=bottom,
  			axis y line*=left, 	
  			xlabel=$z$,
  			ylabel=$E$,
  			axis y line=middle,
			axis x line=middle,
			every axis x label/.style={
			    at={(ticklabel* cs:1.05)},
			    anchor=west,
			},
			every axis y label/.style={
			    at={(ticklabel* cs:1.05)},
			    anchor=south,
			},  			
		]
			\addplot[domain=0:2, samples=100]{x/(1+x^2)^(3/2)};

		\end{axis}
			\draw[dashed] (-0.1,0.385*11.4) node [left] {$E_{max}$} -- ++(5,0);
			\draw[dashed] (2.4,-0.1) node [below] {$z'$} -- ++(0,5);
	\end{tikzpicture}
	\caption{График функции $E(z)$}
	\label{fig:figure1}
\end{figure}

\end{document}