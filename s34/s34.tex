\documentclass[a4paper,14pt]{extarticle}
\def\source{/home/osabio/tex/templates}
\input{\source/head.tex}
\yakovlev{34}{Электростатика}

\begin{document}

\begin{figure}[H]
    \centering
    \begin{tikzpicture}
        \draw[axis,->] (0,0) -- (8,0) node[right] {$+x$};
        \draw[fill=magenta] (0,0) circle (3pt) node[above] {$q$} node[below] {$0$};
        \draw[fill=magenta] (3,0) circle (3pt)node[left] {$P$};
        \draw[] (4,0) circle (1cm) node[above] {$Q$} node[below] {$d$};
    \end{tikzpicture} 
\end{figure}

Сосредоточим весь заряд $Q$ в точке $P$. Тогда потенциал в центре шара найдем как суперпозицию
\begin{equation}
    \phi_o=\phi_q(d)+\phi_Q(R)=\frac{kq}{d}+\frac{kQ}{R}
\end{equation}

Потенциал проводника во всех его точках одинаков, поэтому
\begin{equation}
    \phi_\text{ш}=\phi_o=\frac{kq}{d}+\frac{kQ}{R}
\end{equation}

\end{document}