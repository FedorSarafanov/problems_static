\documentclass[a4paper,14pt]{extarticle}
\def\source{/home/osabio/tex/templates}
\input{\source/head.tex}
\yakovlev{17}{Электростатика}

\begin{document}

Ранее был выведен факт, что в системе двух зарядов поверхностью нулевого потенциала будет сфера. 

Такой системой можно заменить исходную (только при $|x|<r$, так как при $|x|\in[r;R]$ $\vec{E}=0$), заряд $q'$ будет изображением заряда $q$.

\begin{figure}[H]
    \centering
    \begin{tikzpicture}

        % \draw[thick] (0,0) ++ (0,-2) node[below, blue] {$+q$} rectangle ++(0.1,4) ;
        % \draw[thick] (8,0) ++ (0,-2) node[below, red] {$-q$} rectangle ++(-0.1,4);
        \draw[->] (-1,0) -- node[above, xshift=-0.6em] {$r$} ++(50:2);
        \draw[dashed, blue,->] (-3,0) -- (9,0) node[thick,right] {$+x$};
        % \draw[draw=none] (0,4) (0,-4);
        \fill[red] (0,0) circle (4pt) node[above,red] {$q$};
        \fill[blue] (4,0) circle (4pt) node[above,blue] {$q'$};
        % \draw[magenta] (4+0.48*4,0) node[red] {$\times$};
        % \draw[magenta] (-0.18*4,0) node[red] {$\times$};
        \draw[magenta] (-1,0) node[red] {$\times$};
        \draw[magenta,dashed] (-1,0) circle (2cm);
        % \draw[axis] (0,2) -- (4,0) circle (2pt);
        % \lineann[1]{90}{2}{$r$};
        \begin{scope}[xshift=-1cm, yshift=0cm]
            \lineann[-1.2]{0}{5}{$l$};
           	\lineann[-0.7]{0}{1}{$d$};
        \end{scope}
    \end{tikzpicture} 
\end{figure}

Нам известны $r,q,d$. Тогда решим систему, считая равным нулю в точках $x=\pm r$:
\begin{equation}
	\left\{
	\begin{aligned}
 	\frac{kq}{l-r}+\frac{kq'}{r-d}=0\\
 	\frac{kq}{l+r}+\frac{kq'}{r+d}=0
 	\end{aligned}
 	\right. \quad \Rightarrow \quad 
	\left\{
	\begin{aligned}
 	q'=-q\frac{r}{d}\\
 	l=\frac{r^2}{d}
 	\end{aligned}
 	\right. 	
 \end{equation} 

Воспользуемся теоремой Гаусса-Остроградского. Выберем поверхность - цилиндр, одним донышком лежащий на сфере в точке $C$, вторым внутри сферы. Цилиндр должен быть достаточно мал, чтобы поток через его боковую поверхность был равен нулю. 

\begin{figure}[H]
    \centering
    \begin{tikzpicture}

        % \draw[thick] (0,0) ++ (0,-2) node[below, blue] {$+q$} rectangle ++(0.1,4) ;
        % \draw[thick] (8,0) ++ (0,-2) node[below, red] {$-q$} rectangle ++(-0.1,4);
        % \draw[->] (-1,0) -- node[above, xshift=-0.6em] {$r$} ++(50:2);
        \draw (-3,-0.25) rectangle ++ (0.5,0.5) node[above] {$C$};
        \draw (0.5,-0.25) rectangle ++ (0.5,0.5) node[above] {$B$};
        \draw[dashed, blue,->] (-3,0) -- (9,0) node[thick,right] {$+x$};
        % \draw[draw=none] (0,4) (0,-4);
        \fill[red] (0,0) circle (4pt) node[above,red] {$q$};
        \fill[blue] (4,0) circle (4pt) node[above,blue] {$q'$};
        % \draw[magenta] (4+0.48*4,0) node[red] {$\times$};
        % \draw[magenta] (-0.18*4,0) node[red] {$\times$};
        \draw[magenta] (-1,0) node[red] {$\times$};
        \draw[magenta,dashed] (-1,0) circle (2cm);
        % \draw[axis] (0,2) -- (4,0) circle (2pt);
        % \lineann[1]{90}{2}{$r$};
        \begin{scope}[xshift=-1cm, yshift=0cm]
            \lineann[-1.2]{0}{5}{$l$};
           	\lineann[-0.7]{0}{1}{$d$};
        \end{scope}
    \end{tikzpicture} 
\end{figure}
Тогда с левой стороны:
\begin{equation}
	\oiint\limits_{S}(\vec{E},\vec{dS})=
	-\frac{kqS}{(r+d)^2}+
	\frac{k|q'|S}{(r+l)^2}
	=k4\pi q_{in}= -k4\pi\sigma_c\cdot S
\end{equation}
Минус справа здесь, чтобы плотность была модулем (изнутри заряд индуцирован отрицательный).
Откуда нетрудно показать, что
\begin{equation}
	\sigma_c=\frac{q}{4\pi}
	\left[
	\frac{1}{(r+d)^2}-
	\frac{r}{d(r+r^2/d)^2}
	\right]=
	\frac{q}{4\pi(r+d)^2}
	\left[
	1-\frac{d}{r}
	\right]
\end{equation}

Аналогично с правой 
\begin{equation}
	\oiint\limits_{S}(\vec{E},\vec{dS})=
	-
	\frac{kqS}{(r-d)^2}+
	\frac{k|q'|S}{(l-r)^2}
	=k4\pi q_{in2}= -k4\pi\sigma_b\cdot S
\end{equation}
\begin{equation}
	\sigma_b=\frac{q}{4\pi}
	\left[
	\frac{1}{(r-d)^2}-
	\frac{r}{d(r^2/d-r)^2}
	\right]=
	\frac{q}{4\pi(r-d)^2}
	\left[
	1+\frac{d}{r}
	\right]	
\end{equation}

Индуцированный заряд снаружи распределен равномерно, и по модулю равен $q$ (можно показать через т.Гаусса). Тогда
\begin{equation}
	\sigma=\frac{q}{4\pi R^2}
\end{equation}

Потенциал вне такого сферического слоя  будет таким же, как и потенциал точечного заряда $q$ в её центре. Тогда на внешней поверхности:
\begin{equation}
	\phi=\frac{q}{R}
\end{equation}

\end{document}