\documentclass[a4paper,14pt]{extarticle}
\def\source{/home/osabio/tex/templates}
\input{\source/head.tex}
\irodov{3.124}{Электростатика}
\usepackage{mathrsfs}
\begin{document}
\begin{figure}[H]
    \centering
	\begin{circuitikz}
		\draw (0,0)
		to[battery1, l=$\mathscr{E}$] ++(0,2)
		to[closing switch={K}] ++(2,0)
		to[C=$C_2$] ++(0,-2)
		to[] node[inputarrow, rotate=180] {} node[below] {1} ++(-2,0);
		\draw (2,2)
		to[C=$C_1$] ++(2,0);
		% to[battery1=$\mathscr{E}$] ++(0,-2)		
		% to[] ++(-2,0);
		\draw (2,0)
		to node[inputarrow, rotate=0] {} node[below] {2} ++(2,0)

		to[battery1, l_=$\mathscr{E}$] ++(0,2);	
		% to[C=$C_1$] ++(-2,0);		
	\end{circuitikz}
\end{figure}
Найдем заряды на конденсаторах до замыкания ключа K, когда $C_1$ и $C_2$ включены последовательно.
\begin{equation}
	C=\frac{C_1C_2}{C_1+C_2}
\end{equation}
\begin{equation}
	q_1=q_2=q=C\mathscr{E}
\end{equation}
После замыкания 
\begin{equation}
	q'_1=0,
\end{equation}
так как разность потенциалов на обкладках конденсатора равна нулю, и
\begin{equation}
	q'_2=C_2\mathscr{E}
\end{equation}
Через точки 1 и 2 протекает заряд
\begin{equation}
	Q_1+Q_2=q'_2-q_2=C_2\mathscr{E}-C\mathscr{E}
\end{equation}
Но с другой стороны, через точку 2 протек заряд
\begin{equation}
	Q_2=q'_1-q_1=-C\mathscr{E}
\end{equation}
Из двух последних уравнений получим ответ:
\begin{equation}
	Q_1=C_2\mathscr{E}
\end{equation}
\begin{equation}
	Q_2=-\frac{C_1C_2}{C_1+C_2}\mathscr{E}
\end{equation}

\end{document}