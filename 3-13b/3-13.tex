\documentclass[a5paper,10pt]{article}
\def\source{/home/osabio/tex/templates}
\input{\source/head.tex}
\irodovE{3.13b}

\begin{document}

% \begin{tikzpict}
% 	\lineann[1]{0}{2}{$2a$}
% 	\begin{scope}[xshift=1cm]
% 		\lineannn[-1]{0}{0.3}{$dx$}		
% 	\end{scope}
% 	\draw (0,-0.15) rectangle ++(2,0.3);
% 	\draw[fill=magenta] (1,-0.15) rectangle ++(0.3,0.3);
% 	% \draw (1.15,-0.1) node[below] {$dq$};
% 	% \draw (-4,0) circle (0cm);
% 	\vbLabel{0}{-0.1}{0};
% 	\draw[axis,->] (0,0) -- ++(4,0) node[right] {$+x$};

% 	\draw[fill=magenta] (3,0) coordinate (Y) circle (2pt);
% 	\draw[force,->] (Y) -- ++ (0.5,0) node[below] {$d\vec{E}$};
% \end{tikzpict}

\begin{tikzpict}[scale=1.5]
	\lineann[1]{0}{1}{$a$}
	\lineann[2]{0}{1.57}{$x$}
	\begin{scope}[xshift=1.5cm]
		\lineannn[-1]{0}{0.15}{$dx$}		
	\end{scope}
	\begin{scope}[xshift=1cm]
		\lineann[1]{0}{2}{$r$}		
	\end{scope}	
	\begin{scope}[xshift=1.57cm]
		\lineann[2]{0}{1.43}{$R$}		
	\end{scope}		
	\draw (0,-0.15) rectangle ++(2,0.3);
	\draw[fill=magenta] (1.5,-0.15) rectangle ++(0.15,0.3);
	% \draw (1.15,-0.1) node[below] {$dq$};
	% \draw (-4,0) circle (0cm);
	\vbLabel{0}{-0.1}{0};
	\draw[axis,->] (0,0) -- ++(4,0) node[right] {$+x$};

	\draw[fill=magenta] (3,0) coordinate (Y) circle (2pt);
	\draw[force,->] (Y) -- ++ (0.5,0) node[below] {$d\vec{E}$};
\end{tikzpict}

Линейная плотность заряда на стержне
\begin{equation}
	\lambda=\frac{q}{2a}
	\quad\Rightarrow\quad
	dq=\lambda dx
\end{equation}
Напряженность элементарного (точечного) заряда $dq$
\begin{equation}
	dE=\frac{k\cdot \lambda dx}{R^2}
\end{equation}
Причем $R$ - функция (полагаем, что $r>a$):
\begin{equation}
	R=R(x)=(2a-x)+(r-a)=a+r-x
\end{equation}
где $r$ -- расстояние от середины стержня до точки наблюдения, $R$ -- расстояние от произвольного элемента стержня с координатой $x\in[0,2a]$ до точки наблюдения.

% \begin{equation}
% 	dr=-dx
% \end{equation}
Тогда полная напряженность в точке наблюдения
\begin{eqnarray}
	E=
	\int\limits_0^{2a}
	k\frac{\lambda dx}{(a+r-x)^2}=
	k\lambda\int\limits_0^{2a}
	\frac{-d(a+r-x)}{(a+r-x)^2}=
	k\lambda\frac{1}{a+r-x}\bigg|_0^{2a}=\\
	=k\lambda\frac{1}{r-a}-k\lambda\frac{1}{r+a}=
	k\lambda\frac{(r+a)-(r-a)}{r^2-a^2}=k\lambda\frac{2a}{r^2-a^2}
\end{eqnarray}
Так как $2a\lambda=q$, то
\begin{equation}
	E=k\frac{q}{r^2-a^2}
\end{equation}
При $r \gg a$
\begin{equation}
	E=
	k\frac{q}{r^2}
\end{equation}

% Т.е. при большом удалении от стержня его поле становится  полем точеч
\end{document}