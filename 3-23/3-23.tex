\documentclass[a5paper,10pt]{article}
\def\source{/home/osabio/tex/templates}
\input{\source/head.tex}
\irodovE{3.23}
\usepackage{esint}
\begin{document}

\begin{tikzpict}[scale=0.5]
	\draw (0,0) circle (2cm);
	% \filldraw[even odd rule,inner co

	or=black!40,outer color=black!0] (0,0) circle (2);
	\draw[dashed, magenta] (0,0) circle (2.2cm);
	% \draw[dashed, magenta] (0,0) circle (1.8cm);
	% \foreach \s in {0,20,...,360}
	% {
	% 	\draw[->, blue!50] (0,0) ++ (\s:1) -- (\s:2.5);
	% 	\draw[->, blue!50] (0,0) ++ (\s+10:1.5) -- (\s+10:3);
	% }
	\draw[axis,->, black] (0,0) -- (3,0) node[right] {$+r$};
	% \lineann[1]{0}{2}{$2a$}
	% \begin{scope}[xshift=1cm]
	% 	\lineannn[-1]{0}{0.3}{$dx$}		
	% \end{scope}
	% \draw (0,-0.15) rectangle ++(2,0.3);
	% \draw[fill=magenta] (1,-0.15) rectangle ++(0.3,0.3);
	% % \draw (1.15,-0.1) node[below] {$dq$};
	% % \draw (-4,0) circle (0cm);
	% \vbLabel{0}{-0.1}{0};
	% \draw[axis,->] (0,0) -- ++(4,0) node[right] {$+x$};

	% \draw[fill=magenta] (3,0) coordinate (Y) circle (2pt);
	% \draw[force,->] (Y) -- ++ (0.5,0) node[below] {$d\vec{E}$};
\end{tikzpict}
Найдем поле по принципу суперпозиции. Будем считать известным поле шара:
\begin{equation}
	E_s(r)=k\frac{Q}{r^2}
\end{equation}
где Q -- искомый заряд шара.
% \begin{equation}
% 	\rho(r)=\rho_0(1-\frac{r}{R})
% \end{equation}
Запишем теорему Остроградского-Гаусса для сферической поверхности с радиусом $r>R$ (на границе сферического слоя толщиной $r-R$):
\begin{equation}
	\label{eq:og}
	\oiint\limits_{(S)}\vec{E}d\vec{S}=
	k\cdot 4\pi q_{in}
\end{equation}
где
\begin{equation}
	\label{eq:q}
	q_{in}=
	\iiint\limits_{(V)}\rho dV=
	% \left\{
	% \begin{aligned}
		\int\limits_R^r \frac{\alpha}{r} 4\pi r^2 dr=
		4\pi\alpha\int\limits_R^r rdr = 
		4\pi\alpha \frac{r^2}{2}\bigg|_R^r=2\pi\alpha \left[r^2-R^2\right]
		%\\
		% \int\limits_0^R \rho_0(1-\frac{r}{R}) 4\pi r^2 dr=&
		% \frac{1}{3}\pi R^3\rho_0,& \quad if \quad r\geq R
	% \end{aligned}
	% \right.
\end{equation}
С другой стороны,
\begin{equation}
	\label{eq:int}
	\oiint\limits_{(S)}\vec{E}d\vec{S}
		\overset{\vec{E}\parallel \vec{n}}{=}
	\oiint\limits_{(S)}E_rdS
		\overset{E(r)=const}{=}
	E\oiint\limits_{(S)}dS=ES=E\cdot4\pi r^2
\end{equation}
Из формул (\ref{eq:og}), (\ref{eq:q}), (\ref{eq:int}) следует
\begin{equation}
	E(r)=k\frac{2\pi\alpha \left[r^2-R^2\right]}{r^2}=
	k\cdot2\pi\alpha-
	k\cdot2\pi\alpha\frac{R^2}{r^2}
\end{equation}
Тогда полная напряженность в точке наблюдения будет
\begin{equation}
	E_\Sigma=E_s+E=
	k\frac{Q}{r^2}+
	k\cdot2\pi\alpha-
	k\cdot2\pi\alpha\frac{R^2}{r^2}
\end{equation}
Тогда искомый заряд 	
\begin{equation}
	Q=2\pi\alpha R^2
\end{equation}
А постоянная напряженность
\begin{equation}
	E_{const}=k\cdot2\pi\alpha
\end{equation}
\end{document}