\documentclass[a4paper,14pt]{extarticle}
\def\source{/home/osabio/tex/templates}
\input{\source/head.tex}
\irodov{3.113}{Электростатика}
% Стыдить лжеца, шутить над дураком
% И спорить с женщиной — всё то же,
% Что черпать воду решетом:
% От сих троих избавь нас, боже!..

% М. Ю. Лермонтов.
\begin{document}

Воспользуемся методом изображений. В данном случае поле эквивалентно системе двух шариков, симметричных относительно плоскости и противоположно заряженных:

\begin{figure}[H]
    \centering
    \begin{tikzpicture}

        % \draw[thick] (0,0) ++ (0,-2) node[below, blue] {$+q$} rectangle ++(0.1,4) ;
        % \draw[thick] (8,0) ++ (0,-2) node[below, red] {$-q$} rectangle ++(-0.1,4);
        % \draw[->] (-1,0) -- node[above, xshift=-0.6em] {$r$} ++(50:2);
        \draw[dashed, blue,->] (-5,0) -- (5,0) node[thick,right] {$+x$};
        % % \draw[draw=none] (0,4) (0,-4);
        % \fill[red] (0,0) circle (4pt) node[above,red] {$q$};
        % \fill[blue] (4,0) circle (4pt) node[above,blue] {$q'$};
        % % \draw[magenta] (4+0.48*4,0) node[red] {$\times$};
        % % \draw[magenta] (-0.18*4,0) node[red] {$\times$};
        % \draw[magenta] (-1,0) node[red] {$\times$};
        \draw[magenta] (-3,0) circle (1cm);
        \draw[fill=magenta] (-3,0) circle (2pt) node [above] {0} node [below] {$+q$};
        \draw[fill=magenta] (3,0) circle (2pt) node [below] {$-q$};
        \draw[magenta, dashed] (3,0) circle (1cm);
        % \draw[axis] (0,2) -- (4,0) circle (2pt);
        % \lineann[1]{90}{2}{$r$};
        \lineann[-2]{0}{-3}{$l$};
        \lineann[2]{0}{-2}{$l-a$};
        \draw[magenta] (0,-2.5) -- (0,2.5);
    \end{tikzpicture} 
\end{figure}

Напряженность поля на оси, по принципу суперпозиции
\begin{equation}
    E(x)=\frac{kq}{x^2}+\frac{kq}{(2l-x)^2}
\end{equation}
Тогда
\begin{gather}
    \Delta\phi=\int\limits_a^l 
    \left[
    \frac{kq}{x^2}+\frac{kq}{(2l-x)^2}
    \right]dx=
    kq\left[
        \frac{1}{2l-x}
        -
        \frac{1}{x}
    \right]\bigg|_a^l=\\=
    kq\left[\frac{1}{l}-\frac{1}{l}-\frac{1}{2l-a}+\frac{1}{a}\right]
    % kq\left[\frac{1}{a}-\frac{1}{l}-\frac{1}{a-2l}+\frac{1}{l}\right]
    =
    kq\left[\frac{1}{a}-\frac{1}{2l-a}\right]
\end{gather}
\begin{equation}
    \Delta\phi=kq\frac{2l-2a}{2al-a^2}
\end{equation}
Если $l\gg a$, то
\begin{equation}
    \Delta\phi\approx \frac{kq}{a}
\end{equation}
\begin{equation}
    C=\frac{q}{\Delta\phi}=\frac{a}{k}
\end{equation}

\end{document}