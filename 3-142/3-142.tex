\documentclass[a4paper,14pt]{extarticle}
\def\source{/home/osabio/tex/templates}
\input{\source/head.tex}
\irodov{3.142}{Электростатика}
\usepackage{mathrsfs}
\tikzset{
  pics/carc/.style args={#1:#2:#3}{
    code={
      \draw[pic actions] (#1:#3) arc(#1:#2:#3);
    }
  }
}
\begin{document}
\begin{figure}[H]
    \centering
	\begin{circuitikz}		
		\foreach \i in {1.6,1.61,...,2}{
  			\draw pic[thick, magenta]{carc=5:355:\i};
		}
		\draw[fill=magenta] (0,0) circle (2pt);
  		% \draw pic[thick]{carc=5:355:cm};
        \draw[thick,->,dashed] (0,0) -- (4,0) node[right]{$+r$};
        \lineann[4]{90}{2}{$b$};
        \lineann[5]{90}{1.6}{$a$};
	\end{circuitikz}
\end{figure}

В начальный момент поле равно нулю только в объеме проводника.


Когда же мы медленно перенесем заряд в бесконечность, плотность энергии везде будет ненулевой, и тогда вся наша работа пойдет на изменение энергии только в толще проводника:

\begin{equation}
  w(r)=\frac{E^2(r)}{8\pi k}=\frac{k^2q^2}{8\pi k r^4}
\end{equation}
\begin{equation}
  A=\iiint\limits_{(V)} w dV=\int\limits_a^b w(r) 4\pi r^2 dr=\frac{kq^2}{2}\left(\frac{1}{a}-\frac{1}{b}\right)
\end{equation}

\end{document}