\documentclass[a4paper,14pt]{extarticle}
\def\source{/home/osabio/tex/templates}
\input{\source/head.tex}
\yakovlev{21}{Электростатика}

\begin{document}

Решим задачу методом изображений, добавив три заряда: два для изображения плоскостей и третий -- для изображения поверхности нулевого потенциала на дополнениях плоскостей (без него задача несимметрична).

\begin{figure}[H]
    \centering
    \begin{tikzpicture}
        \draw[interface] (0,0) rectangle ++(-0.2,4);
        \draw[interface] (0,0) rectangle ++(4,-0.2);
        \draw[] (0,0) -- ++(0,4);
        \draw[] (0,0) -- ++(4,0);

        \draw[fill=black!80] (0,0) -- (2,2) circle (2pt) node[above] {q};
        \draw[dashed, draw=none] (0,0) -- (-135:{sqrt(8)}) circle (2pt) node[below, draw=white,white] {q};
        \draw[dashed, draw=none] (0,0) -- (135:{sqrt(8)}) circle (2pt) node[below, draw=white,white] {-q};
        \draw[dashed, draw=none] (0,0) -- (-45:{sqrt(8)}) circle (2pt) node[below, draw=white,white] {-q};
        \draw[white] (4,0)--++(2,0);
    \end{tikzpicture} 
    \begin{tikzpicture}
        \draw[interface] (0,0) rectangle ++(-0.2,4);
        \draw[interface] (0,0) rectangle ++(4,-0.2);
        \draw[] (0,0) -- ++(0,4);
        \draw[] (0,0) -- ++(4,0);

        \draw[fill=black!80] (0,0) -- (2,2) coordinate (*) circle (2pt) node[above] {q};
        \draw[dashed] (0,0) -- (-135:{sqrt(8)}) circle (2pt) node[below] {q};
        \draw[dashed] (0,0) -- (135:{sqrt(8)}) circle (2pt) node[below] {-q};
        \draw[dashed] (0,0) -- (-45:{sqrt(8)}) circle (2pt) node[below] {-q};

        \draw[force,->] (*) -- ++(180:1) node [above] {$\vec{F}_1$};
        \draw[force,->] (*) -- ++(-90:1) node [below] {$\vec{F}_2$};
        \draw[force,->] (*) -- ++(45:1) node [below] {$\vec{F}_3$};
    \end{tikzpicture}     
\end{figure}

Cилу будем искать как суперпозицию сил трех зарядов:

\begin{equation}
    \vec{F}=\sum \vec{F}_i=\vec{F}_1+\vec{F}_2+\vec{F}_3
\end{equation}

Расстояние до плоскостей $l=\frac{d}{\sqrt{2}}$. 

Сложение $\vec{F}_2$ и $\vec{F}_1$ в силу симметрии даст вектор, лежаший на оси $d$:
\begin{equation}
    f=|\vec{F_1}+\vec{F_2}|=\sqrt{2}F_1=
    \sqrt{2}\frac{kq^2}{4l^2}=
    \frac{kq^2}{\sqrt{2}d^2}
\end{equation}
Вектор $\vec{F}_3$ по модулю
\begin{equation}
    F_3=\frac{kq^2}{4d^2}
\end{equation}

Тогда найдем суммарную силу:
\begin{equation}
    \vec{F}=-\frac{kq^2\vec{d}}{d^3}\left(\sqrt{2}-\frac12\right)
\end{equation}

\end{document}