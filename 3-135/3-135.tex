\documentclass[a4paper,14pt]{extarticle}
\def\source{/home/osabio/tex/templates}
\input{\source/head.tex}
\irodov{3.135}{Электростатика}
\usepackage{mathrsfs}
\begin{document}
\begin{figure}[H]
    \centering
	\begin{circuitikz}
	 \draw
		(0,0) node[spdt] (Sw) {}
		(Sw.in) node[left] {}
		(Sw.out 1) node[above, xshift=-0.75em] {1}
		(Sw.out 2) node[below, xshift=-0.75em] {2};
		\draw (Sw.in)
		to [C, l_=$C$] ++ (-2,0)
		to ++(0,-2)
		to[battery1, i_=$\mathscr{E}_1$,] ++(4,0) 
		-| coordinate (x) (Sw.out 2);

		\draw (x)++(2.5,0)
		to[battery1, i^=$\mathscr{E}_2$,] ++(-2.5,0);

		\draw (Sw.out 1) -| ($(x)+(2.5,0)$);

	\end{circuitikz}
\end{figure}
Найдем энергию конденсатора при ключе в положении 1, когда $\mathscr{E}_1$ и $\mathscr{E}_2$ включены последовательно.
\begin{equation}
	W_1=\frac{CU^2}{2}=\frac{C(\mathscr{E}_2-\mathscr{E}_1)^2}{2}
\end{equation}
\begin{equation}
	q_1=q_2=q=C\mathscr{E}
\end{equation}
После переключения в положение 2 разность потенциалов на обкладках конденсатора равна $\mathscr{E}_1$
\begin{equation}
	W_2=\frac{C\mathscr{E}_1^2}{2}
\end{equation}
Тогда
\begin{equation}
	W_2-W_1=\frac{C\mathscr{E}_1^2}{2}-\frac{C(\mathscr{E}_2-\mathscr{E}_1)^2}{2}=
	С\mathscr{E}_2\mathscr{E}_1-\frac{C\mathscr{E}_2^2}{2}
\end{equation}
Также нужно учесть работу источника. При переключении ключа через источник протечет заряд $q'$:
\begin{equation}
	q'=q_2-q_1=C\mathscr{E}_1-(C\mathscr{E}_1-C\mathscr{E}_2)=
	C\mathscr{E}_2
\end{equation}
\begin{equation}
	A=\mathscr{E}_1 q'=C\mathscr{E}_2\mathscr{E}_1
\end{equation}
Работа пойдет на выделение тепла и изменение энергии системы:
\begin{equation}
	A=(W_2-W_1)+Q
\end{equation}
Отсюда
\begin{equation}
	Q=A-(W_2-W_1)=\frac{C\mathscr{E}_2^2}{2}
\end{equation}

\end{document}