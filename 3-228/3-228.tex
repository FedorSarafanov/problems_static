\documentclass[a4paper,14pt]{extarticle}
\def\source{/home/osabio/tex/templates}
\input{\source/head.tex}
\irodov{3.228}{Магнитостатика}

\usetikzlibrary{decorations.markings}
\usepackage[outline]{contour}
\contourlength{4pt}
\tikzset{
  pics/carc/.style args={#1:#2:#3}{
    code={
      \draw[pic actions] (#1:#3) arc(#1:#2:#3);
    }
  }
}
\begin{document}

\begin{figure}[H]
    \centering
    \begin{tikzpicture}
    % \begin{scope}[yshift=2cm]
    %     \lineann[2]{0}{-2}{$a$}
    % \end{scope}

    \begin{scope}[
        very thick,decoration={
        markings,
        mark=at position 0.5 with {\arrow{>}}}
        ] 
        % \draw[] (0,0) arc (90:1);
        \draw[thick] (0,0) pic[->]{carc=0:180:4cm};

        \draw (0,0) coordinate (0) circle (2pt) node[below, yshift=-1em]{P};

        \draw[->] (0) -- ++(90:4cm) coordinate (b) node [pos=0.5] {\contour{white}{$\vec{r}$}};

        \draw[->] (0) -- ++(45:4cm) node [pos=0.7] {\contour{white}{$\vec{r}_1$}};

        \draw[->] (0) -- ++(135:4cm) node [pos=0.7] {\contour{white}{$\vec{r}_2$}};

        \draw[->, very thick] (0) -- ++(-135:1.5cm) node [below] {\contour{white}{$d\vec{B}_2$}};        

        \draw[->, very thick] (0) -- ++(135:1.5cm) node [pos=0.5] {\contour{white}{$d\vec{B}_1$}};    


        \draw[axis,->] (0,0) -- (-5,0) node[left] {$+x$};
        % \draw[->] (b) -- ++(180:1.5cm) node [pos=0.5] {\contour{white}{$d\vec{l}$}};
        % \draw[postaction={decorate}] (0,4) coordinate (2)-- node[above] {2}(8,4);
        % \draw[postaction={decorate}] (8,4) coordinate (3)-- node[right] {3}(8,0);
        % \draw[postaction={decorate}] (8,0) coordinate (4)-- node[below] {4}(0,0);

        % \draw[dashed] (1) -- (3);
        % \draw[dashed] (2) -- (4);
        % % \draw[dashed] 
        % %     (-4,4) node[above] {$O$}
        % %     -- (0,0) node[anchor=west] {$A$}
        % %     -- (4,-4) node[right] {$O'$};

        % % \draw[dashed] 
        % %     (4,4) node[above] {$B$}
        % %     -- (-4,-4) node[right] {$B'$};            

        % % \draw (135:{sqrt(2)}) node [] {\contour{white}{$\bigotimes$}}  node[above, yshift=0.5em] {$d\vec{B}_1$};
        % \draw (4,2) node [] {\contour{white}{$\bigotimes$}} node[above, yshift=0.5em] {$d\vec{B}_{1,2,3,4}$};

        % % \draw (45:{sqrt(2)}) node [] {\contour{white}{$\bigodot$}}  node[above, yshift=0.5em] {$d\vec{B}_1$};
        % \draw (0,0) node [] {\contour{white}{$\bigotimes$}} node[right, xshift=0.5em] {$d\vec{B}_2$};     

        \draw (60:4) node [] {\contour{white}{$\bigotimes$}}  node[above, yshift=0.5em] {$\vec{j}$};
        % % \draw (135+90:2.5) node [] {\contour{white}{$\bigodot$}} node[above, yshift=0.5em] {$d\vec{B}_2$};

        % % \draw (45-90:{sqrt(2)}) node [] {\contour{white}{$\bigodot$}}  node[above, yshift=0.5em] {$d\vec{B}_1$};
        % % \draw (45-90:2.5) node [] {\contour{white}{$\bigodot$}} node[above, yshift=0.5em] {$d\vec{B}_2$};   

        % % \draw (5,0) -- (14,0);          
        % % \draw[] (9,-4) -- ++(0,8);          

    \end{scope}    
    \end{tikzpicture}
\end{figure}
Разобьем такой проводник на более простые - линейные проводники. 

Введем плотность тока $\rho=\frac{I}{\pi R}$. Тогда для каждого такого проводника

\begin{equation}
    dB=2\frac{\beta dI}{R}=2\frac{\beta \rho dl}{R}=2\beta \rho d\phi
\end{equation}

\begin{equation}
    dB_x=2\frac{\beta dI}{R}=2\frac{\beta \rho dl}{R}=2\beta \rho  \sin \phi d\phi
\end{equation}

Из соображений симметрии вертикальная компонента поля занулится, а горизонтальную (на оси $+x$) найдем интегрированием по всем элементам разбиения - линейным проводникам:

\begin{equation}
    B=\int\limits_0^{\pi}2\beta \rho \sin \phi d\phi= 4\beta \rho=\frac{4\beta I}{\pi R}=28\text{ мкТл}
\end{equation}

\end{document}